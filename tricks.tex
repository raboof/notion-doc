
\chapter{Advanced scripting, assorted tricks}

This chapter explains some additional possibilities of scripting Ion
that are not part of the basic configuration framework documented in
chapter \ref{chap:config}.

\section{Hooks and other callbacks}

\subsection{Hooks}

Hooks are lists of functions to be called when a certain event occurs.
Hook handlers are registered with the function \fnref{add_to_hook}
and removed with \fnref{remove_from_hook}. Both of these functions
take as argument the name of the hook (a string) and the handler,
the parameters of which depend on the actual hook in question.

The following hooks are currently defined:

\begin{tabularx}{\linewidth}{lX}
\hline
Hook & Description \\
\hline
\code{screen_workspace_switched} &
	Called when the object (not necessarily a workspace despite the
	name) viewed on a screen is switched. Parameters to handler:
	the screen and the newly switched-to region. \\
\code{genframe_managed_switched} &
	Called when the region viewed in a frame is switched. Parameters
	to handler: the frame and the newly switched-to region. \\
\code{deinit} &
	Called when Ion is about to start deinitilising before exiting.
	Handler has no parameters. \\
\end{tabularx}

More hooks can be added on request as need arises.

\subsection{Placement methods}

In addition to the hooks mentioned above there is (at the moment) 
one callback that is not a hook. It is the function
\code{ionws_placement_method} can be used by scripts to decide
in which frame a newly mapped client window should be placed 
within an already decided on \type{WIonWS}. The function has
three parameters: the workspace, the client window and a boolean
indicating whether the client window's geometry (see
\fnref{WRegion.geom}) was specified by the user by e.g. a
\code{-geometry} command line switch. The function should
return a frame on the workspace or \code{nil} if it made no
decision. For example. the window placement heuristics in 
\file{heuristics.lua} implement this function.


%\section{Sample scripts}
%\fname{heuristics.lua}
%\fname{closeorkill.lua}


