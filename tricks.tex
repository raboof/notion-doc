
\chapter{Scripting}
\label{chap:tricks}

This chapter documents some additional features of the Ion configuration
and scripting interface that can be used for more advanced scripting than
the basic configuration explained in chapter \ref{chap:config}.

\section{Hooks}
\label{sec:hooks}

Hooks are lists of functions to be called when a certain event occurs.
There are two types of them; normal and ``alternative'' hooks. Normal
hooks do not return anything, but alt-hooks should return a boolean
indicating whether it handled its assigned task successfully. In the case
that \var{true} is returned, remaining handlers are not called.

Hook handlers are registered by first finding the hook
with \fnref{ioncore.get_hook} and then calling \fnref{WHook.add}
on the (successful) result with the handler as parameter. Similarly
handlers are unregistered with \fnref{WHook.remove}. For example:

\begin{verbatim}
ioncore.get_hook("ioncore_snapshot_hook"):add(
    function() print("Snapshot hook called.") end
)
\end{verbatim}

In this example the hook handler has no parameters, but many hook
handlers do. The types of parameters for each hook are listed in
the hook reference, section \ref{sec:hookref}.

Note that many of the hooks are called in ``protected mode'' and can not 
use any functions that modify Ion's internal state. 


\section{Referring to regions}

\subsection{Direct object references}

All Ion objects are passed to Lua scripts as 'userdatas', and you may
safely store such object references for future use. The C-side object
may be destroyed while Lua still refers to the object. All exported
functions gracefully fail in such a case, but if you need to explicitly
test that the C-side object still exists, use \fnref{obj_exists}.

As an example, the following short piece of code implements 
bookmarking:

\begin{verbatim}
local bookmarks={}

-- Set bookmark bm point to the region reg
function set_bookmark(bm, reg)
    bookmarks[bm]=reg
end

-- Go to bookmark bm
function goto_bookmark(bm)
    if bookmarks[bm] then
        -- We could check that bookmarks[bm] still exists, if we
        -- wanted to avoid an error message.
        bookmarks[bm]:goto()
    end
end
\end{verbatim}

\subsection{Name-based lookups}

If you want to a single non-\type{WClientWin} region with an exact known 
name, use \fnref{ioncore.lookup_region}. If you want a list of all regions,
use \fnref{ioncore.region_list}. Both functions accept an optional argument
that can be used to specify that the returned region(s) must be of a more 
specific type. Client windows live in a different namespace and for them
you should use the equivalent functions \fnref{ioncore.lookup_clientwin}
and \fnref{ioncore.clientwin_list}.

To get the name of an object, use \fnref{WRegion.name}. Please be
aware, that the names of client windows reflect their titles and
are subject to changes. To change the name of a non-client window
region, use \fnref{WRegion.set_name}.


\section{Alternative winprop selection criteria}

It is possible to write more complex winprop selection routines than
those described in section \ref{sec:winprops}. To match a particular
winprop using whatever way you want to, just set the \var{match}
field of the winprop to a function that receives the as its parameters
the triple \var{(prop, cwin, id)}, where \var{prop} is the table for 
the winprop itself, \code{cwin} is the client window object,
and  \var{id} is the \fnref{WClientWin.get_ident} result.
The function should return \var{true} if the winprop matches, 
and \code{false} otherwise. Note that the \var{match} function
is only called after matching against class/role/instance.

The title of a client window can be obtained with \fnref{WRegion.name}.
If you want to match against (almost) arbitrary window properties,
have a look at the documentation for the following functions, and
their standard Xlib counterparts: \fnref{ioncore.x_intern_atom}
(XInternAtom), \fnref{ioncore.x_get_window_property} (XGetWindowProperty),
and \fnref{ioncore.x_get_text_property} (XGetTextProperty).



\section{Writing \command{ion-statusd} monitors}
\label{sec:statusd}

All statusbar meters that do not monitor the internal state of Ion should
go in the separate \command{ion-statusd} program. 

Whenever the user requests a meter \codestr{\%foo} or \codestr{\%foo\_bar} to 
be  inserted in a statusbar, \file{mod\_statusbar} asks \command{ion-statusd} 
to load \fnref{statusd_foo.lua} on its search path (same as that for Ion-side 
scripts). This script should then supply all meters with the initial part
\codestr{foo}.

To provide this value, the script should simply call \code{statusd.inform}
with the name of the meter and the value as a string.
Additionally the script should provide a 'template' for the meter to
facilitate expected width calculation by \file{mod\_statusbar}, and
may provide a 'hint' for colour-coding the value. The interpretation
of hints depends on the graphical style in use, and currently the
stock styles support the \codestr{normal}, \codestr{important} and 
\codestr{critical} hints.


In our example of the 'foo monitor', at script initialisation we might broadcast
the template as follows:

\begin{verbatim}
statusd.inform("foo_template", "000")
\end{verbatim}

To inform \file{mod\_statusbar} of the actual value of the meter and
indicate that the value is critical if above 100, we might write the
following function:

\begin{verbatim}
local function inform_foo(foo)
    statusd.inform("foo", tostring(foo))
    if foo>100 then
        statusd.inform("foo_hint", "critical")
    else
        statusd.inform("foo_hint", "normal")
    end
end    
\end{verbatim}
    
To periodically update the value of the meter, we must use timers.
First we must create one:

\begin{verbatim}
local foo_timer=statusd.create_timer()
\end{verbatim}

Then we write a function to be called whenever the timer expires.
This function must also restart the timer.

\begin{verbatim}
local function update_foo()
    local foo= ... measure foo somehow ...
    inform_foo(foo)
    foo_timer:set(settings.update_interval, update_foo)
end
\end{verbatim}

Finally, at the end of our script we want to do the initial
measurement, and set up timer for further measurements:

\begin{verbatim}
update_foo()
\end{verbatim}


If our scripts supports configurable parameters, the following code
(at the beginning of the script) will allow them to be configured in
\file{cfg\_statusbar.lua} and passed to the status daemon and our script:

\begin{verbatim}
local defaults={
    update_interval=10*1000, -- 10 seconds
}
                
local settings=table.join(statusd.get_config("foo"), defaults)
\end{verbatim}


