
\section{Configuring Ion with Lua}

The following subsections are indended to help users configure Ion
with Lua. If you are new to Lua, you might first want to read the
the Lua manual at \url{http://www.lua.org/docs.html} and perhaps
some helpfull lua-users wiki pages (including a quick to read tutorial)
at \url{http://lua-users.org/wiki/}.

Section \ref{sec:bindings} desribes how keys and mouse
actions are bound to functions and in section \ref{sec:winprops}
winprops are explained. For a reference on exported functions, see
section \ref{sec:exports}.

\subsection{Defining bindings}
\label{sec:bindings}

TODO: describe this more thoroughly

Binding callback functions will receive two arguments: the region
owning the binding and the region where the actual event occured.
For ''normal'' regions like frames these are the same but for e.g.
workspaces that can't directly have bindings but grab them on their
''managed'' frames, the first argument is the workspace and the second
the frame. Also, button event callbacks for frames will receive as the
second argument the owner of the tab where the event occured, if any.

The first parameter to any binding callback function is always of the
type of the respective bindings group (for \fnref{global_bindings}
\type{WScreen}).

As client windows do not have a binding group, it is necessary to
call client window functions by specifying the bindings somewhere
else (e.g. \fnref{global_bindings} or \fnref{ionframe_bindings}
and \fnref{floatframe_bindings}), and then looking up the previously
active leaf node in Ion's hierarchy of objects, starting from the
first parameter to the callback with
\fnref{region_get_active_leaf}%
\index{region-get-active-leaf@\code{region_get_active_leaf}}.
The convenience function \code{make_active_leaf_fn} is provided to
make a function to do this. The following two binding definitions are
equivalent:

\begin{verbatim}
global_bindings {
    kpress("Mod1+C", function(scr)
                         lf=region_get_active_leaf(scr)
                         region_close(lf)
                     end
          ),
    kpress("Mod1+C", make_active_leaf_fn(region_close)),
}
\end{verbatim}

The function
\fnref{genframe_current}
\index{genframe-current@\code{genframe_curent}}
may also be used to get the current managed region of a frame. (The
object returned by this function is not the same as what
\fnref{region_get_active_leaf} would return if there are transients!)

\subsection{Winprops}
\label{sec:winprops}

\subsubsection{Classes, roles and instances}

The so-called ''winprops''\index{winprop} can be used to change how
specific windows are handled and to set up some kludges to deal with
badly behaving applications. They are defined by calling the function
\code{winprop} with a table containing the properties to set and the
necessary information to identify a window. This identification
information is more specifically the
\var{class}\index{class@\var{class}!winprop},
\var{role}\index{role@\var{role}!winprop} and
\var{instance}\index{instance@\var{instance}!winprop}
of the window. It is not necessary to specify all of these; if one is
not specified or is the string \code{"*"} that particular field matches
all windows. For a window with identification information \var{id}, Ion
looks for a matching winprop in the following order:
\begin{verbatim}
for _, c in {id.class, "*"} do
    for _, r in {id.role, "*"} do
        for _, i in {id.instance, "*"} do
            -- Check for winprop matching (c, r, i)
        end
    end
end
\end{verbatim}

To get this identification information for a particular window, you
may use the command \command{xprop WM_CLASS} and click on that
particular window.\footnote{This does not work for transients in
WIonFrames.} The class is the latter of the strings while
the instance is the former. To get the role -- few windows have
this property -- use the command \command{xprop WM_ROLE}.

\subsubsection{Supported winprops}

Ion currently knows the following winprops:

\index{switchto@\var{switchto}!winprop}
\index{transient-mode@\var{transient_mode}!winprop}
\index{target@\var{target}!winprop}
\index{transparent@\var{transparent}!winprop}
\index{acrobatic@\var{acrobatic}!winprop}
\index{max-size@\var{max_size}!winprop}
\index{aspect@\var{aspect}!winprop}
\index{ignore-resizeinc@\var{ignore_resizeinc}}

\begin{tabularx}{\textwidth}{llX}
    \hline
    Property & Type & Description\\\hline
    \var{switchto} &
    	boolean &
    	Should the window be switched to when it is created. \\
    \var{transient_mode} &
  	string &
    	"normal": No change in behaviour. "current": The window
	should be thought of as a transient for the current active
	client window (if any) even if it is not marked as a
	transient by the application. "off": The window should be
	handled as a normal window even if it is marked as a
	transient by the application. \\
    \var{target} &
    	string &
    	The full name of an object (workspace, frame) that should
	manage windows of this type. \\
    \var{transparent} &
    	boolean &
    	Should frames be made transparent when this window is selected. \\
    \var{acrobatic} &
    	boolean &
    	Set this to \code{true} for Acrobat Reader. It has an annoying
	habit of trying to manage its dialogs instead of setting them as
	transients and letting the window manager do its job, causing
	Ion and acrobat go a window-switching loop when a dialog is
	opened. \\
    \var{max_size} &
    	table &
        The table should countain the entries \var{w} and \var{h} that
	override application-supplied maximum size hint. \\
    \var{aspect} &
    	table &
        The table should countain the entries \var{w} and \var{h} that
	override application-supplied aspect ratio hint. \\
    \var{ignore_resizeinc} &
    	boolean &
    	Should application supplied size increments be ignored? \\
\end{tabularx}

\subsubsection{Examples}

Acrobat Reader's manners aren't exactly good:
\begin{verbatim}
winprop{
    class = "AcroRead",
    instance = "documentShell",
    acrobatic = true,
}
\end{verbatim}

Place xterm started with '\code{-name sysmon}' and running a system
monitoring program in a specific frame:
\begin{verbatim}
winprop{
    class = "XTerm",
    instance = "sysmon",
    target = "sysmonframe",
}
\end{verbatim}


\subsection{Miscellaneous issues}

The \code{QueryLib.query_lua} function is quite useless at the moment as it
is not possible to simply enter a name of a command and call it.
The query takes in Lua code and executes it as-is. `\code{arg[1]}' is set
to \myhref{fn:genframe_current}{\code{genframe_current()}} for the frame
in which the query is executing in or to the frame if such does not exist.
This argument can not be directly passed to most functions and it is
necessary to somehow obtain an object of wanted type. Later, a better
handler for executing single commands in addition to to one for executing
arbitrary code should be written, in Lua.

